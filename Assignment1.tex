\documentclass[a4paper,12pt]{article}
\usepackage{tfrupee}
\usepackage{amsmath}
\title{ASSIGNMENT-1}
\author{SENANI SADHU}
\date{\today}
\begin{document}
	\maketitle
	\pagenumbering{roman}
	\section{Question:-}
	\paragraph{The bookshop of a particular school has
		10 dozen chemistry books, 8 dozen physics
		books, 10 dozen economics books. Their
		selling prices are \rupee 80, \rupee 60 and \rupee 40 each
		respectively. Find the total amount the
		bookshop will receive from selling all the
		books using matrix algebra.}
	\section{Solution:-}
	No of chemistry books =10 dozen= 10*12 = 120
	\newline
	No of physics books = 8 dozen = 8*12 = 96
	\newline
	No of economics books = 10 dozen = 10*12=120
	\newline
	Let matrix A denote the no of books per subject.
	\newline
	\begin{equation}
		A=\begin{bmatrix}
			120 &96 & 120
		  \end{bmatrix}
	\end{equation}
	\newline
	Let matrix B  denote  the selling price of  books  per subject.
	\newline
	\begin{equation}
		B=\begin{bmatrix}
			80\\
			60\\
			40
		\end{bmatrix}
	\end{equation}
	 \newline
	 Now,
	 \newline
	 Total amount the shopkeeper receives =
	 No of books * Selling Price = AB
	 =\begin{equation}
	 	\begin{bmatrix}
	 		120 & 96 &120
	 	\end{bmatrix}*\begin{bmatrix}
	 	80\\
	 	60\\
	 	40
 	\end{bmatrix}
	 \end{equation}
 \newline
 =\begin{equation}
 	\begin{bmatrix}
 		120*80 + 96*60 + 120*40
 	\end{bmatrix}
 \end{equation}
\newline
=120*80 + 96*60 + 120*40= 9600 + 5760 + 4800=\paragraph{20160}

\paragraph{Hence,total amount received by shopkeeper is \rupee20160}
\end{document}